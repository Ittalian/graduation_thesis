\chapter{考察}
\label{ch:eval}

\quad

アノテーションの定義と,アノテーションの内部データ構造を説明する~\cite{flanagan2008ruby}.

\section{ドキュメントごとの結果の解釈}
\label{sec:eval_xxx}

モゲという概念を用いる.モゲという概念を用いる.モゲという概念を用いる~\cite{goto2010bioruby}.
モゲという概念を用いる.モゲという概念を用いる.モゲという概念を用いる.
モゲという概念を用いる.モゲという概念を用いる.モゲという概念を用いる.
モゲという概念を用いる~\cite{matsumoto2002ruby}.モゲという概念を用いる.モゲという概念を用いる.
モゲという概念を用いる.モゲという概念を用いる.モゲという概念を用いる.
モゲという概念を用いる.モゲという概念を用いる.モゲという概念を用いる.


\begin{itemize}
 \item 1991 年の×△△△△の AAA は秀逸であると感じた。
 \item 2001 年の△△△△△の BBB がよかった。
 \item 1997 年の△◯×の CCC がややよかった。
\end{itemize}

箇条書きは便利ですが、○○なので控える。
論理的なごまかしや妥協につながる可能性があるかもしれないかもしれないと思
うのかもしれない。表や地の文で書く。


モゲという概念を用いる.モゲという概念を用いる.モゲという概念を用いる.
モゲという概念を用いる.モゲという概念を用いる.モゲという概念を用いる.
モゲという概念を用いる.モゲという概念を用いる.モゲという概念を用いる.

\section{改善点}
\label{sec:eval_yyy}

模気という構造を用いる.模気という構造を用いる.模気という構造を用いる~\cite{richardson2008restful}.
模気という構造を用いる.模気という構造を用いる.模気という構造を用いる.
模気という構造を用いる.模気という構造を用いる.模気という構造を用いる.
模気という構造を用いる.模気という構造を用いる.模気という構造を用いる.
模気という構造を用いる.模気という構造を用いる.模気という構造を用いる.
模気という構造を用いる.模気という構造を用いる.模気という構造を用いる.
模気という構造を用いる.模気という構造を用いる.模気という構造を用いる.
模気という構造を用いる.模気という構造を用いる.模気という構造を用いる.

模気という構造を用いる.模気という構造を用いる.模気という構造を用いる.
模気という構造を用いる.模気という構造を用いる.模気という構造を用いる.
模気という構造を用いる.模気という構造を用いる.模気という構造を用いる~\cite{sasada2005yarv}.
模気という構造を用いる.模気という構造を用いる.模気という構造を用いる.
模気という構造を用いる.模気という構造を用いる.模気という構造を用いる.
模気という構造を用いる.模気という構造を用いる.模気という構造を用いる.
模気という構造を用いる.模気という構造を用いる.模気という構造を用いる.
模気という構造を用いる.模気という構造を用いる.模気という構造を用いる.


