\chapter{考察}
\label{ch:eval}

\quad

\section{ドキュメントごとの結果の解釈}
\label{sec:eval_docs}

第~\ref{ch:exp}章で本研究で実施した実験を紹介したが,本章ではその結果から,本研究で提示したシステムの優位点及び将来性や有用性について述べる.
まず,実験により明らかとなった本システムの優位点について述べる.それは,ドキュメントの形式に関わらず同様の処理結果を得ることができる点である.
文書には,縦書きや横書き及びそれらの複合など,内容の形式はジャンルによって多岐にわたる.それら全てに対応出来なければ,本システムの優位性は著しく低下してしまう.
しかし,さまざまな形式の文書を実験では用意したが,特に形式の差による精度の変化は見受けられなかった.そのため,本システムにおいては,文書をアップロードする前に形式を変換する必要がなく,
その点ではストレスなくシステムを運用することができると考えられる.本システムの将来性については,発展途上であるという結論になってしまうが,将来的にカテゴリが増え,解析の幅が広がることや,
WordNet データベースがアップデートされ,扱うことのできる語彙が増えていくことを考えると,IT 分野の発展に伴い,相対的にニーズが増えていくと考えられる.
本システムの有用性については,第~\ref{ch:intro}章で大まかに述べたが,紙媒体中心の業務に不便を感じている人が現代に多くいるため,そのような状況を改善するという観点では,要件を満たすことができているといえる.

\section{改善点}
\label{sec:eval_improve}
本システムの改善点について挙げられることは,WordNet データベースに存在しない専門的な単語を多く含む文書に関しては,精度が落ちてしまう点である.
この欠点を改善するためには,現状使用している WordNet データべースの他に,各々のカテゴリに即した専門的なデータベースを使用し,判別可能な語彙を拡張することが必要である.
内容の異なる複数のデータベースを1つの MVC アーキテクチャに統合することができれば,本システムの有用性や汎用性の向上につながると考えられる.
また,現状のシステムでは,日本語と英語を主な対象としており,WordNet データベースを利用してカテゴリの解析をしている.しかし,グローバル化が進む現代社会では,多言語対応が求められる場面が増えている.
特に,アジア地域で広く使用されている中国語,韓国語に対応することで,システムの利用範囲をさらに拡大することが可能である.
これを実現するためには,各言語に特化した語彙データベースを統合し,多言語間でのカテゴリ解析を可能にするアルゴリズムの開発が必要である.
