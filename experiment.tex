\chapter{実験 ・ 評価}
\label{ch:exp}
\quad

\section{様々なドキュメントでの実験}
\label{sec:exp_doc}

本研究の実験で用意するカテゴリは,経理,人事,事務の3種類であり,各カテゴリごとに5種類ずつの計15種類のドキュメントで行う.
\begin{enumerate}
    \item \textbf{経理}
        \subitem{会社経理統制と経理検査}
        \subitem{工業経理規範}
        \subitem{戦時会社経理統制体制の展開}
        \subitem{法人税法の損金経理要件について}
        \subitem{陸軍経理組織の変遷と内部監査制度}
    \item \textbf{人事}
        \subitem{トヨタウェイと人事管理・労使管理}
        \subitem{公務員の人事異動と人材形成}
        \subitem{新・人事労務管理}
        \subitem{人事労務管理}
        \subitem{戦略的パートナーとしての日本の人事部}
    \item \textbf{事務}
        \subitem{ロボティックス・プロセス・オートメーションが事務職に及ぼす影響に関する一考察}
        \subitem{一般事務女性の職業生活意識に関する一考察}
        \subitem{女性事務職に得る派遣労働者の活用}
        \subitem{女性事務職のキャリア拡大と職場組織}
        \subitem{女性事務職の賃金と就業行動}
\end{enumerate}

\section{精度の分析}
\label{sec:exp_anal}

\subsection{タイトルごとの精度}
\label{subsec:title}

さまざまなフォーマットや内容の文書を対象にOCR処理を実施し、タイトルごとの処理精度を評価した。その結果を表\ref{tab:title_accuracy}に示す。

\begin{table}[h]
  \centering
  \begin{tabular}{|l|c|l|}
    \hline
    \textbf{タイトル} & \textbf{結果} & \textbf{備考} \\ \hline
    会社経理統制と経理検査 & 失敗 & 昔の字体が存在する \\ \hline
    工業経理規範 & 失敗 & 字体は全て昔の書き方 \\ \hline
    戦時会社経理統制体制の展開 & 成功 & 縦書き2列構成 \\ \hline
    法人税法の損金経理要件について & 成功 & 減価償却等の専門的な情報 \\ \hline
    陸軍経理組織の変遷と内部監査制度 & 成功 & \\ \hline
    トヨタウェイと人事管理・労使管理 & 成功 & 縦書き2列構成 \\ \hline
    公務員の人事異動と人材形成 & 成功 & 図や英語が混在 \\ \hline
    新・人事労務管理 & 成功 & フォントがゴシック体の太文字 \\ \hline
    人事労務管理 & 成功 & 縦書きと横書きが混在 \\ \hline
    戦略的パートナーとしての日本の人事部 & 成功 & 図や見出しが英語表記 \\ \hline
    RPA が事務職に及ぼす影響に関する一考察 & 成功 & \\ \hline
    一般事務女性の職業生活意識に関する一考察 & 成功 & 見出しのみ英語表記 \\ \hline
    女性事務職に得る派遣労働者の活用 & 成功 & 縦書き2列構成のゴシック体 \\ \hline
    女性事務職のキャリア拡大と職場組織 & 成功 & \\ \hline
    女性事務職の賃金と就業行動 & 成功 & \\ \hline
  \end{tabular}
  \caption{タイトルごとの精度}
  \label{tab:title_accuracy}
\end{table}

表\ref{tab:title_accuracy}に示したように、処理精度にはタイトルごとにばらつきが見られる。特に、「会社経理統制と経理検査」や「工業経理規範」のように、旧字体を含む文書では処理が失敗する傾向が確認された。一方で、縦書きや横書きが混在する文書、英語表記が一部含まれる文書、ゴシック体や太文字を用いた文書などにおいては、ほとんど問題なく処理が成功している。これは、現代の標準的なフォーマットに近い文書に対しては、OCRの処理が適切に機能することを示している。
また、縦書き2列構成や図表が含まれる場合も成功率は高く、特に人事関連のタイトルにおいては安定した処理精度が見られた。これらの結果から、フォーマットや字体の違いがOCRの精度に与える影響が明確になった。

\clearpage

\subsection{カテゴリごとの精度}
\label{subsec:category}

続いて、文書をカテゴリに分類し、それぞれのカテゴリごとに処理精度を評価した結果を表\ref{tab:category_accuracy}に示す。

\begin{table}[h]
  \centering
  \begin{tabular}{|l|c|}
    \hline
    \textbf{カテゴリ} & \textbf{精度 (\%)} \\ \hline
    経理 & 60 \\ \hline
    人事 & 100 \\ \hline
    事務 & 100 \\ \hline
  \end{tabular}
  \caption{カテゴリごとの精度}
  \label{tab:category_accuracy}
\end{table}

表\ref{tab:category_accuracy}に示したように、経理カテゴリにおける精度が他のカテゴリと比べて著しく低いことが確認された。この主な要因としては、経理カテゴリの文書に旧字体や特殊な専門用語が多く含まれている点が挙げられる。一方で、人事や事務カテゴリにおいては、フォーマットが比較的一般的であるため、100\%の精度を達成している。
さらに、精度が高いカテゴリにおいても、処理内容には軽微な誤解釈が存在する場合があるが、全体としてカテゴリごとの分類精度に影響を与えるような問題は確認されなかった。これにより、OCR処理のカテゴリごとの安定性について一定の評価を下すことができる。
以上の結果から、タイトルごとの処理精度とカテゴリごとの精度において、それぞれの特性が確認された。本研究では、特定の条件下における処理の成功率を明らかにしたが、さらなる精度向上を目指すためには、旧字体や特殊なフォーマットへの対応が課題として残る。これらの詳細な考察については次章で述べる。
