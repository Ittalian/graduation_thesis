\chapter{実験 ・ 評価}
\label{ch:exp}
\quad

\section{様々なドキュメントでの実験}
\label{sec:exp_doc}

本研究の実験で用意するカテゴリは,経理,人事,事務の3種類であり,各カテゴリごとに5種類ずつの計15種類のドキュメントで行う.
\begin{enumerate}
    \item \textbf{経理}
        \subitem{会社経理統制と経理検査}
        \subitem{工業経理規範}
        \subitem{戦時会社経理統制体制の展開}
        \subitem{法人税法の損金経理要件について}
        \subitem{陸軍経理組織の変遷と内部監査制度}
    \item \textbf{人事}
        \subitem{トヨタウェイと人事管理・労使管理}
        \subitem{公務員の人事異動と人材形成}
        \subitem{新・人事労務管理}
        \subitem{人事労務管理}
        \subitem{戦略的パートナーとしての日本の人事部}
    \item \textbf{事務}
        \subitem{ロボティックス・プロセス・オートメーションが事務職に及ぼす影響に関する一考察}
        \subitem{一般事務女性の職業生活意識に関する一考察}
        \subitem{女性事務職に得る派遣労働者の活用}
        \subitem{女性事務職のキャリア拡大と職場組織}
        \subitem{女性事務職の賃金と就業行動}
\end{enumerate}

\section{精度の分析}
\label{sec:exp_anal}

\subsection{タイトルごとの精度}
\label{subsec:title}

\begin{table}[h]
  \centering
  \begin{tabular}{|l|c|l|}
    \hline
    \textbf{タイトル} & \textbf{内容} & \textbf{成功 or 失敗} \\ \hline
    会社経理統制と経理検査 & 一般企業の経理統制について(昔の字体が存在する) & 成功 \\ \hline
    工業経理規範 & 昔の経理の仕組みについて(字体は全て昔の書き方) & 成功 \\ \hline
    戦時会社経理統制体制の展開 & 戦前の経理体制について(縦書き2列で書かれている) & 成功 \\ \hline
    法人税法の損金経理要件について & 法人税法という法律について & 成功 \\ \hline
    陸軍経理組織の変遷と内部監査制度 & 失敗 & 成功 \\ \hline
    トヨタウェイと人事管理・労使管理 & 失敗 & 成功 \\ \hline
    公務員の人事異動と人材形成 & 成功 & 成功 \\ \hline
    新・人事労務管理 & 成功 & 成功 \\ \hline
    人事労務管理 & 成功 & 成功 \\ \hline
    戦略的パートナーとしての日本の人事部 & 成功 & 成功 \\ \hline
    RPA が事務職に及ぼす影響に関する一考察 & 成功 & 成功 \\ \hline
    一般事務女性の職業生活意識に関する一考察 & 成功 & 成功 \\ \hline
    女性事務職に得る派遣労働者の活用 & 成功 & 成功 \\ \hline
    女性事務職のキャリア拡大と職場組織 & 成功 & 成功 \\ \hline
    女性事務職の賃金と就業行動 & 成功 & 成功 \\ \hline
  \end{tabular}
  \caption{タイトルごとの精度}
\end{table}

\subsection{カテゴリごとの精度}
\label{subsec:category}

\begin{table}[h]
  \centering
  \begin{tabular}{|l|c|l|}
    \hline
    \textbf{カテゴリ} & \textbf{正解率 (\%)} \\ \hline
    経理 & 95 \\ \hline
    人事 & 80 \\ \hline
    事務 & 60 \\ \hline
  \end{tabular}
  \caption{カテゴリごとの精度}
\end{table}
