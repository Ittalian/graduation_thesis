\chapter{実験 ・ 評価}
\label{ch:exp}

\quad

アノテーションの定義と,アノテーションの内部データ構造を説明する~\cite{flanagan2008ruby}.

\section{様々なドキュメントでの実験}
\label{sec:exp_xxx}

モゲという概念を用いる.モゲという概念を用いる.モゲという概念を用いる~\cite{goto2010bioruby}.
モゲという概念を用いる.モゲという概念を用いる.モゲという概念を用いる.
モゲという概念を用いる.モゲという概念を用いる.モゲという概念を用いる.
モゲという概念を用いる~\cite{matsumoto2002ruby}.モゲという概念を用いる.モゲという概念を用いる.
モゲという概念を用いる.モゲという概念を用いる.モゲという概念を用いる.
モゲという概念を用いる.モゲという概念を用いる.モゲという概念を用いる.
モゲという概念を用いる.モゲという概念を用いる.モゲという概念を用いる.
モゲという概念を用いる.モゲという概念を用いる.モゲという概念を用いる.
モゲという概念を用いる.モゲという概念を用いる~\cite{fulton2006ruby}.

% setsumei
メタな記号
$\$$, $\/$

% setsumei
数式

$\sum_{k=0}^{100} (\lfloor x^k \rfloor + 1)$

モゲという概念を用いる.モゲという概念を用いる.モゲという概念を用いる.
モゲという概念を用いる.モゲという概念を用いる.モゲという概念を用いる.
モゲという概念を用いる.モゲという概念を用いる.モゲという概念を用いる.
モゲという概念を用いる.モゲという概念を用いる.モゲという概念を用いる.
モゲという概念を用いる.モゲという概念を用いる.

\section{精度の分析}
\label{subsec:exp_yyy}

模気という構造を用いる.模気という構造を用いる.模気という構造を用いる~\cite{richardson2008restful}.
模気という構造を用いる.模気という構造を用いる.模気という構造を用いる.
模気という構造を用いる.模気という構造を用いる.模気という構造を用いる.
模気という構造を用いる.模気という構造を用いる.模気という構造を用いる.
模気という構造を用いる.模気という構造を用いる.模気という構造を用いる.
模気という構造を用いる.模気という構造を用いる.模気という構造を用いる.
模気という構造を用いる.模気という構造を用いる.模気という構造を用いる.
模気という構造を用いる.模気という構造を用いる.模気という構造を用いる.

模気という構造を用いる.模気という構造を用いる.模気という構造を用いる.
模気という構造を用いる.模気という構造を用いる.模気という構造を用いる.
模気という構造を用いる.模気という構造を用いる.模気という構造を用いる~\cite{sasada2005yarv}.
模気という構造を用いる.模気という構造を用いる.模気という構造を用いる.
模気という構造を用いる.模気という構造を用いる.模気という構造を用いる.
模気という構造を用いる.模気という構造を用いる.模気という構造を用いる.
模気という構造を用いる.模気という構造を用いる.模気という構造を用いる.
模気という構造を用いる.模気という構造を用いる.模気という構造を用いる.

\section{既存手法との比較}
\label{subsec:exp_zzz}

模気という構造を用いる.模気という構造を用いる.模気という構造を用いる.
模気という構造を用いる.模気という構造を用いる.模気という構造を用いる.
模気という構造を用いる.模気という構造を用いる.模気という構造を用いる.
模気という構造を用いる.模気という構造を用いる.模気という構造を用いる~\cite{richardson2008restful}.

% setsumei
\ref{exp_xxx} 節 で述べた◯◯は黒い。
特に \ref{exp_yyy }項 のはとても黒い。
\ref{exp_zzz}項 のはやや黒い。