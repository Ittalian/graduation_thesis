\chapter{関連技術}
\label{ch:rt}

\quad

\section{基礎技術}
\label{sec:rw_basic}

\subsection{Ruby}
\label{subsec:rw_sub_ruby}
Ruby~\cite{ruby2023}~\cite{rubybook2024}は,1993 年にまつもとゆきひろ (松本行弘,通称 Matz) が日本で開発したオブジェクト指向言語である.
名前はまつもとの同僚の誕生石であるルビーが由来となっている.本研究では,システムのバックエンドでのテーブル操作に使用する.

\subsection{Ruby on {R}ails}
\label{subsec:rw_sub_rails}
デンマークのプログラマであるデイヴィッド・ハイネマイヤー・ハンソン(通称DHH)氏によって作られた.
エンジニアの間では略称であるRailsまたはRoRと呼ばれることも多く,簡単なコードでWebアプリケーションの開発ができるように設計されている~\cite{takahashi2013rails}.
本研究では,フロントエンドの ERB ファイルを始めとした,Web システムとしての実装に使用する.

\subsection{O{C}{R} ({O}ptional {C}haracter {R}ecognition)}
印刷された文字や手書きの文字などをカメラやスキャナといった光学的な手段でデータとして取り込み,それを解読(文字認識)することによって一度印刷されてしまった文字をパソコンなどのコンピューターが利用できる文字(テキスト)データに変換する技術.
データ入力作業の手間を大幅に削減し2重入力や人的ミスの削減などを目的とした利用はビジネス用途にも広く浸透しており,流通・製造・医療・小売などあらゆる業界で本来は人が読むために印刷された文字をコンピューターに取り込みたいといった要望が根強くあり,バーコードや2次元コードが普及した現在でも需要はむしろ高まる傾向にあります.

\subsection{Google {C}loud {V}ision {API}}
\label{subsec:rw_sub_ocr}
Google が提供する画像認識サービスのことである.Google 独自の機械学習モデルを採用しており,効率的に画像を分析し,
オブジェクトや顔の検出や手書き文字の読み取り,有用な画像メタデータの構築など様々なことを実現できる~\cite{ohnishi2018ocr}.
本研究では,ドキュメントの画像に対して OCR 処理をして,単語ごとに分類するために使用する.

\subsection{Word{N}et}
\label{subsec:rw_sub_wordnet}
大規模な語彙データベースのこと.名詞,動詞,形容詞,副詞は,認知同義語(Synset)のセットにグループ化され,それぞれが異なる概念を表現する.
概念は,概念的意味的および語彙的関係によって相互にリンクされている.
テーブル構造としては,概念テーブル(Synset), 語彙テーブル(Word)とそのリレーションテーブル(Sense)によって形成されている~\cite{takeuchi2019wordnet}.
本研究では,単語を該当するカテゴリに分け,ラベリングをするために使用する.

\section{クラスタリングの既存手法}
\label{sec:rw_tech_sub}

本研究では,あくまでカテゴライズという目的でシステムを構築しているため,クラスタリング手法とは少し異なる点があるが,アルゴリズム等の観点で参考にしたため,関連技術として以下に示す.

\subsection{非階層的クラスタリング}
\label{subsec:rw_non_hierar}

目的のデータを事前に定義されたクラスタ数に分解することによって処理されるクラスタリング方式のことである.
代表的な手法として,クラスタ内の分散を最小化するようにデータポイントをグループ化する k-means アルゴリズムが挙げられる~\cite{kurahashi2007k-means}.

\subsection{階層的クラスタリング}
\label{sebsec:rw_hierar}

データポイントを機構造の階層に分割して処理するクラスタリング方式のことである.
代表的な手法には大きく分けて,凝集型と分割型の2種類がある.
凝集型は,木構造の下から上へクラスタを統合していく方法である.対して分割型は,上から下へクラスタを分割していく方法である.
データの類似度に基づいて徐々にクラスタを形成していくため,クラスタ数を事前に決定する必要がなく,
クラスタの適切な数が不明瞭な場合や,データの階層的な構造を理解したい場合には分割型が有用である~\cite{2023hierar}.
