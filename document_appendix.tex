\chapter*{付録B\\実験で使用したドキュメント}

\setlength{\parindent}{0pt}

\section*{経理}
久保田秀樹, 「会社経理統制令と経理検査」, 山内隆教授退官記念論文集, 第329号, pp. 169-172, 1934年
\\\\
大野巌, 「工業経理規範」, 昭和13年9月, 化学機械, 第2巻 第3号, 1938年
\\\\
柴田雅, 「戦時会社経理統制体制の展開」, The Socio-Economic History Society, 1937年
\\\\
前原真一, 『法人税法の損金経理要件について』, 税務大学校, 研究部教授
\\\\
建部宏明, 「陸軍経理組織の変遷と内部監査制度Ⅱ」, 明治大学経理研究所, 2009年
\\\\
土山真由美, 「企業におけるERPシステム導入の課題と取り組み」, 開発工学 Vol.41 No.2, 2021年
\\\\
黒木淳, 「経営者のアスピレーションの欠如と管理会計の実践度」, 会計プログレス 第21号, 2020年
\\\\
渡邉公認会計士税理士事務所, 「中堅・中小企業のための実践的内部統制の進め方のポイント総点検」, 内部統制報告書, 2008年
\\\\
山田恵, 「わが国法人中小企業の会計情報システムに関する実証的研究」, 管理会計学 Vol.8 No.1-2, 2000年
\\\\
独立行政法人 中小企業基盤整備機構, 「中小企業経営者の経営情報の収集・活用に関する実態調査」, 中小機構調査研究報告書 第5巻 第7号, 2013年

\clearpage
\section*{人事}
猿田正機, 「トヨタウェイと人事管理・労使関係」, 税務経理協会, 2007年
\\\\
圓生和之, 「公務員の人事異動と人材形成―大卒ホワイトカラーの公民比較からの分析」, 日本労働研究雑誌第759号, pp. 47-53, 2023年
\\\\
津田眞澂, 「新・人事労務管理」, 有斐閣, 1995年
\\\\
津田眞澂編著, 「人事労務管理」, ミネルヴァ書房,1993年
\\\\
平野光俊, 「戦略的パートナーとしての日本の人事部―その役割の本質と課題―」, 神戸大学大学院経営学研究科,2010年
\\\\
産労総合研究所, 「第8回 人事制度等に関する総合調査」, 2021年
\\\\
関田良彦, 「中小企業における従業員の満足度とモチベーションに関する調査分析」, 高知工科大学マネジメント学部研究報告, 2017年
\\\\
山本和史, 「中小企業における新卒採用行動に関する実証分析」, 日本労務学会誌 Vol.18 No.1, 2017年
\\\\
佐藤俊一, 「中小企業における人材育成(企業内研修の事例研究)」, 日本経営診断学会第53回全国大会, 2020年
\\\\
柿沼英樹, 「日本企業におけるタレントマネジメントの展開と現状」, Works Discussion Paper No.4, 2015年

\clearpage
\section*{処務}
所正文, 「一般事務職女性の職業生活意識に関する一考察」, 経営行動科学第3巻第1号, 1988年
\\\\
古武真美, 「女性事務職における派遣労働者の活用」, 近畿大学短期大学論集第44巻第1号, pp. 11-20, 2011年
\\\\
浅海典子, 「女性事務職のキャリア拡大と職場組織」, 日本経済評論社, 2006年
\\\\
寺村絵里子,「女性事務職の賃金と就業行動」,国際短期大学人口学研究第48号,2012年
\\\\
日高義浩,「RPA が事務職に及ぼす影響に関する一考察」,鹿児島経済論集第63巻第1号,2022年

\clearpage
\section*{営業}
高彰培, 「CRM主要成功要因と成果間の関連性に対する実証的研究」, NAIS Journal, 2022年
\\\\
松尾睦, 早川勝夫, 高嶋克義, 「改善志向の営業プロセス管理 ― 日本ベーリンガーインゲルハイムの事例」, マーケティングジャーナル Vol.30 No.3, 2011年
\\\\
小須田庸平, 「企業の営業力向上についての考察 ― インターナルマーケティングアプローチの視点から」, 企業研究報告, 2022年
\\\\
橋本佳子, 「中小企業経営改善のための営業組織強化手法」, 中小企業診断報告書, 2006年
\\\\
中小企業庁, 「中小企業実態基本調査」, 令和5年度, 2023年

\clearpage
\section*{エンジニア}
今仁武臣, 中野冠, 「アジャイル型開発手法の適用領域とプロジェクトの成功度の関係」, 日本情報経営学会誌 Vol. 37 No. 150, 2017年
\\\\
小室睦, 鷲崎弘宣, 「ソフトウェアプロセス改善を組織的、実証的にすすめるためのデータ分析パターン言語の提案」, ソフトウェア品質シンポジウム 2024, 2024年
\\\\
小笠原秀人, 藤巻昇, 艸薙匠, 田原康之, 大須賀昭彦, 「大規模組織におけるソフトウェアプロセス改善活動の適用評価―10年間の実践に基づく考察」, 情報処理学会論文誌 Vol. 51 No. 9, 2010年
\\\\
濱﨑利之, 水本継, 横山晃生, 「地方中小企業におけるソフトウェア品質改善の取り組み」, NDKCOM, 2024年
\\\\
井上博進, 金沢健, 「中小企業におけるIT活用の実態と課題」, 愛知工業大学 研究報告, 1999年
\\\\
田中宏和, 「中小企業を対象にしたアジャイル開発の進め方」, 静岡大学, 2015年
\\\\
保田洋, 「中小企業のプロジェクトマネジメントに対する課題解決のためのリスク分析」, 甲子園短期大学紀要 37, 2019年
\\\\
保田洋, 川向肇, 西村治彦, 「中小企業プロジェクトの円滑なマネジメントに向けた失敗要因の分析」, 情報知識学会誌 Vol. 30 No. 3, 2020年
\\\\
保田洋, 川向肇, 西村治彦, 「中小企業プロジェクトの失敗回避に向けたリスク分析」, 第28回年次大会予稿, 2020年
\\\\
野田哲夫, 丹生晃隆, 「日本のIT企業におけるオープンソース・ソフトウェアの活用・開発貢献が企業成長に与える影響に関する研究」, 経済科学論集 第42号, 2016年
