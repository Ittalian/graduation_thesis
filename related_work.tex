\chapter{関連研究}
\label{ch:rw}

\quad

\section{基礎技術}
\label{sec:rw_basic}

\subsection{Ruby}
\label{subsec:rw_sub_ruby}
Ruby~\cite{ruby2023}~\cite{rubybook2024}は,1993 年にまつもとゆきひろ (松本行弘,通称 Matz) が日本で開発したオブジェクト指向言語である.
名前はまつもとの同僚の誕生石であるルビーが由来となっている.本研究では,システムのバックエンドでのテーブル操作に使用する.

\subsection{Ruby on {R}ails}
\label{subsec:rw_sub_rails}
デンマークのプログラマであるデイヴィッド・ハイネマイヤー・ハンソン(通称DHH)氏によって作られた.
エンジニアの間では略称であるRailsまたはRoRと呼ばれることも多く,簡単なコードでWebアプリケーションの開発ができるように設計されている~\cite{takahashi2013rails}.
本研究では,フロントエンドの ERB ファイルを始めとした,Web システムとしての実装に使用する.

\subsection{Google {C}loud {V}ision {API}}
\label{subsec:rw_sub_ocr}
Google が提供する画像認識サービスのことである.Google 独自の機械学習モデルを採用しており,効率的に画像を分析し,
オブジェクトや顔の検出や手書き文字の読み取り,有用な画像メタデータの構築など様々なことを実現できる~\cite{ohnishi2018ocr}.
本研究では,ドキュメントの画像に対して OCR を行い,単語ごとに分類するために使用する.

\subsection{Word{N}et}
\label{subsec:rw_sub_wordnet}
大規模な語彙データベースのこと.名詞,動詞,形容詞,副詞は,認知同義語(Synset)のセットにグループ化され,それぞれが異なる概念を表現する.
概念は,概念的意味的および語彙的関係によって相互にリンクされている.
テーブル構造としては,概念テーブル(Sense), 語彙テーブル(Word)とそのリレーションテーブル(Synset)によって形成されている~\cite{takeuchi2019wordnet}.
本研究では,単語を該当するカテゴリに分け,ラベリングを行うために使用する.

\section{クラスタリングの既存手法}
\label{sec:rw_tech_sub}

\subsection{非階層的クラスタリング}
\label{subsec:rw_non_hierar}

目的のデータを事前に定義されたクラスタ数に分解することによって行われるクラスタリング方式のことである.
代表的な手法として,クラスタ内の分散を最小化するようにデータポイントをグループ化する k-means アルゴリズムが挙げられる~\cite{kurahashi2007k-means}.

\subsection{階層的クラスタリング}
\label{sebsec:rw_hierar}

データポイントを機構造の階層に分割して行うクラスタリング方式のことである.
代表的な手法には大きく分けて,凝集型と分割型の2種類がある.
凝集型は,木構造の下から上へクラスタを統合していく方法である.対して分割型は,上から下へクラスタを分割していく方法である.
データの類似度に基づいて徐々にクラスタを形成していくため,クラスタ数を事前に決定する必要がないため,
クラスタの適切な数が不明瞭な場合や,データの階層的な構造を理解したい場合には分割型が有用である~\cite{2023hierar}.

\section{クラスタリングの課題}
\label{sec:rw_prob}

クラスタリングの大きな課題として挙げられるのは,適切なクラスタ数の決定である.特に非階層的アルゴリズムの場合は,クラスタの数を事前に定義する必要があるが,これが直感的に明らかでない場合が非常に多い.
クラスタリングは,データの中から自動的にパターンや構造を見つけ出す,教師なし学習~\cite{unsupervised}と呼ばれる手法を使用しているため,生成されたクラスタリングの解釈が主観的になりやすい.
言い換えれば,クラスタリングの結果をどう捉えるかがこちらに委ねられている.
