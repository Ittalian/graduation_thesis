\chapter{関連研究}
\label{ch:rw}

\quad

本章では、クラスタリングの手法として有名な手法や課題、使用する基礎技術を示す。

\section{既存手法やクラスタリングの課題}
\label{sec:rw_xxx}

モゲという概念を用いる.モゲという概念を用いる.モゲという概念を用いる~\cite{goto2010bioruby}.
モゲという概念を用いる.モゲという概念を用いる.モゲという概念を用いる.
モゲという概念を用いる.モゲという概念を用いる.モゲという概念を用いる.
モゲという概念を用いる~\cite{matsumoto2002ruby}.モゲという概念を用いる.モゲという概念を用いる.
モゲという概念を用いる.モゲという概念を用いる.モゲという概念を用いる.
モゲという概念を用いる.モゲという概念を用いる.モゲという概念を用いる.
モゲという概念を用いる.モゲという概念を用いる.モゲという概念を用いる.
モゲという概念を用いる.モゲという概念を用いる.モゲという概念を用いる.
モゲという概念を用いる.モゲという概念を用いる~\cite{fulton2006ruby}.

% setsumei
メタな記号
$\$$, $\/$

% setsumei
数式

$\sum_{k=0}^{100} (\lfloor x^k \rfloor + 1)$

モゲという概念を用いる.モゲという概念を用いる.モゲという概念を用いる.
モゲという概念を用いる.モゲという概念を用いる.モゲという概念を用いる.
モゲという概念を用いる.モゲという概念を用いる.モゲという概念を用いる.
モゲという概念を用いる.モゲという概念を用いる.モゲという概念を用いる.
モゲという概念を用いる.モゲという概念を用いる.

\section{基礎技術}
\label{subsec:rw_yyy}

模気という構造を用いる.模気という構造を用いる.模気という構造を用いる~\cite{richardson2008restful}.
模気という構造を用いる.模気という構造を用いる.模気という構造を用いる.
模気という構造を用いる.模気という構造を用いる.模気という構造を用いる.
模気という構造を用いる.模気という構造を用いる.模気という構造を用いる.
模気という構造を用いる.模気という構造を用いる.模気という構造を用いる.
模気という構造を用いる.模気という構造を用いる.模気という構造を用いる.
模気という構造を用いる.模気という構造を用いる.模気という構造を用いる.
模気という構造を用いる.模気という構造を用いる.模気という構造を用いる.

模気という構造を用いる.模気という構造を用いる.模気という構造を用いる.
模気という構造を用いる.模気という構造を用いる.模気という構造を用いる.
模気という構造を用いる.模気という構造を用いる.模気という構造を用いる~\cite{sasada2005yarv}.
模気という構造を用いる.模気という構造を用いる.模気という構造を用いる.
模気という構造を用いる.模気という構造を用いる.模気という構造を用いる.
模気という構造を用いる.模気という構造を用いる.模気という構造を用いる.
模気という構造を用いる.模気という構造を用いる.模気という構造を用いる.
模気という構造を用いる.模気という構造を用いる.模気という構造を用いる.

% setsumei
\ref{rw_xxx} 節 で述べた◯◯は黒い。
特に \ref{rw_yyy }項 のはとても黒い。
