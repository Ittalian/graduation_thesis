\chapter{関連研究}
\label{ch:rw}

\quad

\section{RPA に関する研究}
\label{sec:rpa}

\subsection{RPA の歴史}
\label{subsec:rpa_history}

Madakam ら~\cite{Madakam2019}は,RPA の技術的発展とその適用範囲について調査している.RPA がどのように進化してきたか,また,どのような業務プロセスに適用されてきたかについて述べている.

2010年代前半の RPA は,単純な定型業務の自動化を目的として開発され,そのシンプルな仕組みが特徴であった.
しかし,2018年に本格的な業務への展開が加速し,実際に金融,ヘルスケア,製造業,行政機関など,さまざまな業界で採用が広がり,year of Robotics Process Automation (RPA の年)と称されるようになった.
また,近年では AI (人工知能)や機械学習との統合により,より高度な業務への適用が可能になっている.そのため,企業が競争力を維持するために RPA 導入を急速に進めていることが示されている.

特に,RPA の適用事例として,人事業務,経理処理,請求書管理,データ移行などのバックオフィス業務における導入効果を詳細に分析している.
これらの業務は,本研究で開発するシステムの適用対象とも重なるため,本研究の有用性を裏付ける重要な知見となる.

\subsection{RPA の将来と可能性}
\label{subsec:rpa_future}

Lee ら~\cite{lee2023}は,RPA が雇用に与える影響について詳細に論じている.彼らの研究では,RPA が企業の業務効率化やコスト削減に寄与する一方で,人間の雇用に対する影響についても議論されている.
具体的には,RPA の導入が進むことで,反復的なルーチンタスクを自動化し,生産性の向上やヒューマンエラーの削減,24時間365日の稼働等のメリットをもたらすと述べられている.
一方で,RPA の普及により,一部の業務が自動化されることで雇用の減少につながる可能性も指摘されている.

また,RPA が最も影響を与える業界や職種を特定し,企業の適応戦略についても分析している.特に,RPA の導入によって,ホワイトカラー業務のあり方が変化していくことを強調している.
これらの知見は,RPA の導入に対して企業がどのように向き合うべきかを示唆するものであり,本研究の目的と関連性が高い.

\section{カテゴライズに関する先行研究}
\label{sec:categolize}

\subsection{機械学習を用いたカテゴライズ}
\label{subec:cate_study}

花房ら~\cite{hanabusa2015}は,プログラミング読解中の視線軌道を機械学習によりカテゴライズすることを試みた.従来,プログラミング技能の評価は主観的な評価や定性的な分析が中心であった.しかし,視線追跡データを用い,
教師あり学習を用いた学習モデルの1つである SVM (Support Vector Machine)を活用することにより,技能の異なる学習者(得意群・普通群・不得意群)の視線軌道パターンの定量的に分類した.
大学4年生24名を対象とし,C 言語のプログラムを提示しながら視線の動きを計測した.ヒートマップデータを簡略化し,視線パターンの特徴を抽出した.

任意のデータを定量的に分類し,精度を分析している点は,本研究と共通している.主観的評価に依存せず客観的な分析を実現している点は,機械学習によるカテゴライズの利点といえるだろう.
一方で,低難易度の問題では分類精度が低下したという結果から,解析結果の信頼性が,使用するデータセットに依存してしまう点は,カテゴライズの課題点だといえる.

\subsection{ドキュメントのカテゴライズ}
\label{subsec:cate_doc}

松田ら~\cite{matsuda1999}は,Web ページ(HTML 文書)を,HTML タグと,その要素の情報によってカテゴライズすることで,WWW 検索システムの拡張を試みた.従来の検索の仕組みは,
検索キーワードと全てのページを照合し,一致率の高い順番で表示する.しかし,あらかじめ,Web ページをいくつかのカテゴリに分類しておくことで,検索速度や,内容一致度の向上が見込まれる.

処理の具体的な流れを示す.カテゴライズの基準となる特徴記述ファイルを用意する.ファイル内には,タグと要素の組み合わせごとに,どのカテゴリに分類されるかの詳細な条件が記述されている.
読み込んだ HTML ファイルと特徴記述ファイルを比較し,文書のカテゴリを決定する.

文章に対して内容を解析し,カテゴライズするという点が本研究と共通している.特に解析方法について,Web ページ全体を複数の要素に分割し,各々に対して処理をした後に,
文書全体のカテゴリを判別するという流れは,本研究でも取り入れている.

\section{先行研究との比較}
\label{sec:rela}

\subsection{先行研究との類似点}
\label{subsec:same}

第~\ref{sec:categolize}節で示した2つの先行研究と同様に,特定の媒体に対してカテゴライズを実施しており,主観的な判断を排除しデータに基づく定量的な結果の分析を重視している.
特に松田らが実施した,カテゴライズに関する研究~\cite{matsuda1999}では,文書を要素ごとに分割し,それぞれのカテゴリを判別する手法を採用しているのに対し,本研究でも文書を単語レベルに分割し,各単語の定義を解析することで文書全体のカテゴリを決定するアプローチを採用している.

\subsection{先行研究との差異}
\label{subsec:diff}

一方で,これらの研究との大きな違いは,システムのコストパフォーマンスと拡張性である.花房らが実施した,視線誘導に関する研究~\cite{hanabusa2015}では,機械学習モデルを使用しているが,
このような高い精度を担保するモデルを1つの処理ごとに使用すると,時間的,金銭的にも高コストになってしまう.本研究では,そのような機械学習モデルは使用しておらず.少ないリソースで構築可能なシステムになっている.
さらに,拡張性に関しても,機械学習モデルを使用している場合はそのモデルにある程度依存した結果しか得られないが,本研究ではカテゴライズの条件を自由にカスタマイズできるため,各々の企業に最適なシステムを追求することができる.

また,松田らが実施した,文書タイプのカテゴライズに関する研究~\cite{matsuda1999}では,自身でカテゴライズの基準となるファイルを作成する必要があり.カテゴリが増えるたびに判別条件を考える必要がある.本研究では,単語の定義からカテゴライズするため,
カテゴリに該当する単語を容易に取得することができ,システムの構築や拡張段階において,コストを低く抑えることができる.
