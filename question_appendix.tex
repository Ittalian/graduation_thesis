
\chapter*{付録A\\質疑応答}

\subsubsection*{大原先生の質問(要約)}
業務文書はタイトルでキーワードが選別でき,数サンプル与えれば機械学習でカテゴライズ出来てしまうが,あえてWordNetを使用する理由はなにか.
\subsubsection*{当日の回答}
タイトルや内容が曖昧なドキュメントも,文章全体を見て正しくカデゴライズするために,WordNetを使用している.また,コストの面で機械学習モデルを用いた関連研究との差別化を行うために,カテゴライズのアルゴリズムを自ら考える必要があると判断した.

\subsubsection*{森田先生の質問(要約)}
1つの単語に複数の定義が所属しているが,単語の多義性の部分で何か行っていることはあるか.
\subsubsection*{当日の回答}
単語の定義に該当するものを全てカテゴライズの指標として使用しており,カテゴリに紐づく定義を含む単語をキーワードとして使用している.

\subsubsection*{森田先生の質問(要約)}
文脈によって意味が変わる単語も存在するか,定義の取得はどのようにしているか.
\subsubsection*{当日の回答}
今回単語はカテゴリ名に付随しており,キーワードという形でアルゴリズムに適用しているため,文脈によって変わるような難しい単語ではないと判断している.

\subsubsection*{シュデシナ先生の質問(要約)}
紙媒体が好きな人もいるが,このシステムを使うユーザーはどのような人か.
\subsubsection*{当日の回答}
紙媒体を使用しているが,IT知識の乏しさによってデジタル化したくてもできない企業が主なターゲットになっている.
