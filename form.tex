\documentclass[dvipdfmx,uplatex,12pt,a4j]{ujreport}

\usepackage{./sty/TitlePage}
\usepackage[dvipdfmx]{graphicx}
\usepackage[T1]{fontenc}
\usepackage{textcomp}
\usepackage[utf8]{inputenc}
\usepackage[uplatex]{otf}
\usepackage{listings,jlisting} % ソースコード読み込み用
\usepackage{color,url}
\usepackage{cite}
\usepackage{listings}
\usepackage{tikz}
\usepackage{graphicx}
\usepackage{booktabs}
\usepackage{xcolor}
\usepackage{pdfpages}

\usetikzlibrary{positioning, shapes.geometric, arrows}

\lstset{
    basicstyle=\ttfamily\footnotesize,
    keywordstyle=\color{blue}\bfseries,
    stringstyle=\color{red},
    commentstyle=\color{green},
    frame=single,
    breaklines=true,
    numbers=left,
    numberstyle=\tiny,
    backgroundcolor=\color{white},
    tabsize=2,
    captionpos=b,
}

\tikzstyle{entity} = [rectangle, draw, minimum width=4cm, minimum height=1cm, text centered, node distance=2cm]
\tikzstyle{attribute} = [ellipse, draw, text centered, minimum width=3cm, font=\small]
\tikzstyle{relationship} = [diamond, draw, text centered, minimum width=2cm, font=\small]
\tikzstyle{line} = [draw, -latex]

\setlength {\oddsidemargin}{10.4mm}
\setlength {\evensidemargin}{10.4mm}
\baselineskip 17pt plus 1pt minus 1pt

\newcommand{\lw}[1]{\smash{\lower2.0ex \hbox{#1}}} % 表の中で位置を下にずらして書く

\makeatletter
\def\ngram{{\it n}-gram}
\makeatother

\ronbun{~}


% ユーザー定義の色 (各自適宜変更)
\definecolor{OliveGreen}{cmyk}{0.64,0,0.95,0.40}
\definecolor{Pink}{rgb}{1,0.07,0.54}
\definecolor{CadetBlue}{cmyk}{0.62,0.57,0.23,0}
\definecolor{Brown}{cmyk}{0,0.81,1,0.60}
\definecolor{LightBrown}{rgb}{0.63,0.44,0}
\definecolor{Red}{rgb}{0.95, 0.35, 0.35}
\definecolor{Blue}{rgb}{0.20, 0.35, 0.85}


% ソースコード読み込み用 (各自適宜変更)
\lstset{
	language=C,
	basicstyle={\ttfamily\normalsize},
	identifierstyle={\small},
    % identifierstyle={\color{LightBrown}\small},
    commentstyle={\color{Brown}\normalsize\slshape},
	%keywordstyle={\normalsize\bfseries},
    keywordstyle={\color{Blue}\normalsize\bfseries},
    keywordstyle={[2]\color{Red}},        % Function
    keywordstyle={[3]\color{CadetBlue}},
	ndkeywordstyle={\color{LightBrown}\normalsize},
	stringstyle={\color{Pink}\normalsize},
    backgroundcolor={\color[gray]{.98}},
	frame={tb},
	breaklines=true,
	columns=[l]{fullflexible},
	numbers=left,
	xrightmargin=0zw,
	xleftmargin=3zw,
	% numberstyle={\normalsize},
    numberstyle={\ttfamily\small},
	stepnumber=1,
	numbersep=1zw,
	lineskip=-0.5ex,
	showstringspaces=false,
	tabsize=2,
	captionpos=b,
}

\shozoku{青山学院大学理工学部\\情報テクノロジー学科D\"{u}rst研究室}
\title{OCR とラベリングによる\\書類整理自動化システムの構築と\\有用性の評価}
\author{嘉松 一汰}
\studentID{学生番号:15820094}
\date{令和6年度}

\begin{document}

\includepdf{./cover.pdf}
\includepdf[page=1]{./abstract.pdf}

\maketitle
\pagenumbering{roman}

\tableofcontents

\clearpage\setcounter{page}{1}\pagenumbering{arabic}

\input{./introduction.tex}
\input{./related_tech.tex}
\chapter{関連研究}
\label{ch:rw}

\quad

\section{RPA に関する研究}
\label{sec:rpa}

\subsection{RPA の歴史}
\label{subsec:rpa_history}

Madakam ら~\cite{Madakam2019}は,RPA の技術的発展とその適用範囲について調査している.RPA がどのように進化してきたか,また,どのような業務プロセスに適用されてきたかについて概観している.
RPA は,当初,単純な定型業務の自動化を目的として開発されたが,近年では AI (人工知能)や機械学習との統合が進み,より高度な業務への適用が可能になっている.
また,金融,ヘルスケア,製造業,行政機関など,さまざまな業界で RPA が活用されていることが示されている.

特に,RPA の適用事例として,人事業務,経理処理,請求書管理,データ移行などのバックオフィス業務における導入効果を詳細に分析している.
これらの業務は,本研究で開発するシステムの適用対象とも重なるため,本研究の有用性を裏付ける重要な知見となる.

\subsection{RPA の将来と可能性}
\label{subsec:rpa_future}

Lee ら~\cite{lee2023}は,RPA が雇用に与える影響について詳細に論じている.彼らの研究では,RPA が企業の業務効率化やコスト削減に寄与する一方で,人間の雇用に対する影響についても議論されている.
具体的には,RPA の導入が進むことで,反復的なルーチンタスクを自動化し,生産性の向上やミスの削減,24時間365日の稼働といったメリットをもたらすと述べられている.
しかしながら,RPA の普及によって,一部の業務が自動化されることで雇用の削減につながる可能性も指摘されている.

また,RPA が最も影響を与える業界や職種を特定し,企業の適応戦略についても分析している.特に,RPA の導入によってホワイトカラー業務のあり方が変化し,新たなスキル習得の必要性が高まっている点を強調している.
これらの知見は,RPA の導入に際して企業がどのような対策を講じるべきかを示唆するものであり,本研究の目的と関連性が高い.

\section{カテゴライズに関する先行研究}
\label{sec:categolize}

\subsection{機械学習を用いたカテゴライズ}
\label{subec:cate_study}

花房ら~\cite{hanabusa2015}は,プログラミング読解中の視線軌道を機械学習によりカテゴライズすることを試みた.従来,プログラミング技能の評価は主観的な評価や定性的な分析が中心であった.しかし,視線追跡データを用い,
教師あり学習を用いた学習モデルの1つである SVM (Support Vector Machine)を活用することにより,技能の異なる学習者(得意群・普通群・不得意群)の視線軌道パターンの定量的に分類した.
大学4年生24名を対象とし,C 言語のプログラムを提示しながら視線の動きを計測した.ヒートマップデータを簡略化し,視線パターンの特徴を抽出した.

任意のデータを定量的に分類し,精度を分析している点は,本研究と共通している.主観的評価に依存せず客観的な分析を実現している点は,機械学習によるカテゴライズの利点といえるだろう.
一方で,低難易度の問題では分類精度が低下したという結果から,解析結果の信頼性が,使用するデータセットに依存してしまう点は,カテゴライズの課題点だといえる.

\subsection{ドキュメントのカテゴライズ}
\label{subsec:cate_doc}

松田ら~\cite{matsuda1999}は,Web ページ(HTML 文書)を,HTML タグと,その要素の情報によってカテゴライズすることで,WWW 検索システムの拡張を試みた.従来の検索の仕組みは,
検索キーワードと全てのページを照合し,一致率の高い順番で表示する.しかし,あらかじめ,Web ページをいくつかのカテゴリに分類しておくことで,検索速度や,内容一致度の向上が見込まれる.

処理の具体的な流れを示す.カテゴライズの基準となる特徴記述ファイルを用意する.ファイル内には,タグと要素の組み合わせごとに,どのカテゴリに分類されるかの詳細な条件が記述されている.
読み込んだ HTML ファイルと特徴記述ファイルを比較し,文書のカテゴリを決定する.

文章に対して内容を解析し,カテゴライズするという点が本研究と共通している.特に解析方法について,Web ページ全体を複数の要素に分割し,各々に対して処理をした後に,
文書全体のカテゴリを判別するという流れは,本研究でも取り入れている.

\section{先行研究との比較}
\label{sec:rela}

\subsection{先行研究との類似点}
\label{subsec:same}

第~\ref{sec:categolize}節で示した2つの先行研究と同様に,特定の媒体に対してカテゴライズを実施しており,主観的な判断を排除しデータに基づく定量的な結果の分析を重視している.
特に松田らが実施した,カテゴライズに関する研究~\cite{matsuda1999}では,文書を要素ごとに分割し,それぞれのカテゴリを判別する手法を採用しているのに対し,本研究でも文書を単語レベルに分割し,各単語の定義を解析することで文書全体のカテゴリを決定するアプローチを採用している.

\subsection{先行研究との差異}
\label{subsec:diff}

一方で,これらの研究との大きな違いは,システムのコストパフォーマンスと拡張性である.花房らが実施した,視線誘導に関する研究~\cite{hanabusa2015}では,機械学習モデルを使用しているが,
このような高い精度を担保するモデルを1つの処理ごとに使用すると,時間的,金銭的にも高コストになってしまう.本研究では,そのような機械学習モデルは使用しておらず.少ないリソースで構築可能なシステムになっている.
さらに,拡張性に関しても,機械学習モデルを使用している場合はそのモデルにある程度依存した結果しか得られないが,本研究ではカテゴライズの条件を自由にカスタマイズできるため,各々の企業に最適なシステムを追求することができる.

また,松田らが実施した,文書タイプのカテゴライズに関する研究~\cite{matsuda1999}では,自身でカテゴライズの基準となるファイルを作成する必要があり.カテゴリが増えるたびに判別条件を考える必要がある.本研究では,単語の定義からカテゴライズするため,
カテゴリに該当する単語を容易に取得することができ,システムの構築や拡張段階において,コストを低く抑えることができる.

\input{./approach.tex}
\chapter{実験 ・ 評価}
\label{ch:exp}
\quad

\section{様々なドキュメントでの実験}
\label{sec:exp_doc}

本研究の実験で用意するカテゴリは,経理,人事,事務の3種類であり,各カテゴリごとに5種類ずつの計15種類のドキュメントで行う.
\begin{enumerate}
    \item \textbf{経理}
        \subitem{会社経理統制と経理検査}
        \subitem{工業経理規範}
        \subitem{戦時会社経理統制体制の展開}
        \subitem{法人税法の損金経理要件について}
        \subitem{陸軍経理組織の変遷と内部監査制度}
    \item \textbf{人事}
        \subitem{トヨタウェイと人事管理・労使管理}
        \subitem{公務員の人事異動と人材形成}
        \subitem{新・人事労務管理}
        \subitem{人事労務管理}
        \subitem{戦略的パートナーとしての日本の人事部}
    \item \textbf{事務}
        \subitem{ロボティックス・プロセス・オートメーションが事務職に及ぼす影響に関する一考察}
        \subitem{一般事務女性の職業生活意識に関する一考察}
        \subitem{女性事務職に得る派遣労働者の活用}
        \subitem{女性事務職のキャリア拡大と職場組織}
        \subitem{女性事務職の賃金と就業行動}
\end{enumerate}

\section{精度の分析}
\label{sec:exp_anal}

\subsection{タイトルごとの精度}
\label{subsec:title}

\begin{table}[h]
  \centering
  \begin{tabular}{|l|c|l|}
    \hline
    \textbf{タイトル} & \textbf{内容} & \textbf{成功 or 失敗} \\ \hline
    会社経理統制と経理検査 & 一般企業の経理統制について(昔の字体が存在する) & 成功 \\ \hline
    工業経理規範 & 昔の経理の仕組みについて(字体は全て昔の書き方) & 成功 \\ \hline
    戦時会社経理統制体制の展開 & 戦前の経理体制について(縦書き2列で書かれている) & 成功 \\ \hline
    法人税法の損金経理要件について & 法人税法という法律について & 成功 \\ \hline
    陸軍経理組織の変遷と内部監査制度 & 失敗 & 成功 \\ \hline
    トヨタウェイと人事管理・労使管理 & 失敗 & 成功 \\ \hline
    公務員の人事異動と人材形成 & 成功 & 成功 \\ \hline
    新・人事労務管理 & 成功 & 成功 \\ \hline
    人事労務管理 & 成功 & 成功 \\ \hline
    戦略的パートナーとしての日本の人事部 & 成功 & 成功 \\ \hline
    RPA が事務職に及ぼす影響に関する一考察 & 成功 & 成功 \\ \hline
    一般事務女性の職業生活意識に関する一考察 & 成功 & 成功 \\ \hline
    女性事務職に得る派遣労働者の活用 & 成功 & 成功 \\ \hline
    女性事務職のキャリア拡大と職場組織 & 成功 & 成功 \\ \hline
    女性事務職の賃金と就業行動 & 成功 & 成功 \\ \hline
  \end{tabular}
  \caption{タイトルごとの精度}
\end{table}

\subsection{カテゴリごとの精度}
\label{subsec:category}

\begin{table}[h]
  \centering
  \begin{tabular}{|l|c|l|}
    \hline
    \textbf{カテゴリ} & \textbf{正解率 (\%)} \\ \hline
    経理 & 95 \\ \hline
    人事 & 80 \\ \hline
    事務 & 60 \\ \hline
  \end{tabular}
  \caption{カテゴリごとの精度}
\end{table}

\chapter{考察}
\label{ch:eval}

\quad

\section{ドキュメントごとの結果の解釈}
\label{sec:eval_docs}

第~\ref{ch:exp}で本研究で実施した実験を紹介したが,本章ではその結果から,本研究で提示したシステムの優位点及び将来性や有用性について述べる.
まず,実験により明らかとなった本システムの優位点について述べる.それは,ドキュメントの形式に関わらず同様の処理結果を得ることができる点である.
文書には,縦書きや横書き及びそれらの複合など,内容の形式はジャンルによって多岐にわたる.それら全てに対応出来なければ,本システムの優位性は著しく低下してしまう.
しかし,さまざまな形式の文書を実験では用意したが,特に形式の差による精度の変化は見受けられなかった.そのため,本システムにおいては,文書をアップロードする前に形式を変換する必要がなく,
その点ではストレスなくシステムを運用することができると考えられる.本システムの将来性については,発展途上であるという結論になってしまうが,将来的にカテゴリが増え,解析の幅が広がることや,
WordNet データベースがアップデートされ,扱うことのできる語彙が増えていくことを考えると,IT 分野の発展に伴い,相対的にニーズが増えていくと考えられる.
本システムの有用性については,第~\ref{ch:intro}章で大まかに述べたが,紙媒体中心の業務に不便を感じている人が現代に多くいるため,そのような状況を改善するという観点では,要件を満たすことができているといえる.

\section{改善点}
\label{sec:eval_improve}
本システムの改善点について挙げられることは,WordNet データベースに存在しない専門的な単語を多く含む文書に関しては,精度が落ちてしまう点である.
この欠点を改善するためには,現状使用している WordNet データべースの他に,各々のカテゴリに即した専門的なデータベースを使用し,判別可能な語彙を拡張することが必要である.
内容の異なる複数のデータベースを1つの MVC アーキテクチャに統合することができれば,本システムの有用性や汎用性の向上につながると考えられる.
また,現状のシステムでは,日本語と英語を主な対象としており,WordNet データベースを利用してカテゴリの解析をしている.しかし,グローバル化が進む現代社会では,多言語対応が求められる場面が増えている.
特に,アジア地域で広く使用されている中国語,韓国語に対応することで,システムの利用範囲をさらに拡大することが可能である.
これを実現するためには,各言語に特化した語彙データベースを統合し,多言語間でのカテゴリ解析を可能にするアルゴリズムの開発が必要である.

\chapter{おわりに}
\label{ch:con}

\quad

\section{研究のまとめ}
\label{sec:con_fin}

本研究で開発したシステムは,現在の業務フローにおける課題を解決し得る有効なツールであると結論付けられる.また,さらなる改良を通じて,専門分野や多言語対応,クラウドサービス化といった新たな領域への応用が期待される.
現代では,今後も IT 分野の著しい成長に遅れをとった企業を中心に,さまざまな業務的ニーズが発生すると考えられる.本研究では紙媒体中心業務をさらに細分化した,
ドキュメントのカテゴライズ分野に着目してシステムの作成をしたが,本システムのようなソリューションが生み出されることで,IT 社会の課題が解決されることを願っている.

\section{今後の展望}
\label{sec:con_future}

本システムを現実的な業務フローに組み込むには,ユーザーインターフェース(UI)の最適化も重要である.現状ではシステムのコア機能に重点を置いているが,操作の直感性やユーザーエクスペリエンスの向上が求められる.
例えば,解析結果を可視化するダッシュボードや,フィードバック機能を備えたインターフェースを実装することで,利用者の利便性を高めることが可能である.
処理の精度について,経理,人事,処務といったカテゴリにおいて,高い精度での解析が可能であることが確認された.本システムはさらに分野ごとに特化した最適化を施すことで,
専門分野への応用を推進できる,例えば医療分野や法律分野では特有の専門用語や文書構造が存在するため,分野別にカスタマイズされた解析アルゴリズムが必要である.
また,本システムをクラウドベースで提供することにより,複数のユーザーが同時に利用できる環境を構築し,スケーラビリティを確保することができる.また,APIを公開することで,他のアプリケーションやサービスとの統合を可能にし,システムの拡張性をさらに高めることも有効である.


\newpage
\addcontentsline{toc}{chapter}{謝辞}
\input{thanks.tex}

\newpage
\addcontentsline{toc}{chapter}{参考文献}
\renewcommand{\bibname}{参考文献}
\bibliographystyle{junsrt}
\bibliography{biblio}

\newpage
\addcontentsline{toc}{chapter}{付録}
\thispagestyle{empty}
\vspace*{5.0truecm}
\begin{center}
  {\Huge 付録}
\end{center}
\vspace*{5.0truecm}
{\Large
A. 質疑応答
}

\newpage
\addtocontents{toc}{\vspace{1em}}
\addtocontents{toc}{\noindent\textbf{付録 A}\par}
\addtocontents{toc}{\noindent\hspace{4em} 質疑応答 \hfill \textbf{A-1}\par}

\renewcommand{\thepage}{A-\arabic{page}}
\setcounter{page}{1}

\chapter*{付録A\\質疑応答}

\subsubsection*{大原先生の質問(要約)}
業務文書はタイトルでキーワードが選別でき,数サンプル与えれば機械学習でカテゴライズ出来てしまうが,あえてWordNetを使用する理由はなにか.
\subsubsection*{当日の回答}
タイトルや内容が曖昧なドキュメントも,文章全体を見て正しくカデゴライズするために,WordNetを使用している.また,コストの面で機械学習モデルを用いた関連研究との差別化を行うために,カテゴライズのアルゴリズムを自ら考える必要があると判断した.

\subsubsection*{森田先生の質問(要約)}
1つの単語に複数の定義が所属しているが,単語の多義性の部分で何か行っていることはあるか.
\subsubsection*{当日の回答}
単語の定義に該当するものを全てカテゴライズの判別基準として使用しており,カテゴリに紐づく定義を含む単語をキーワードとして使用している.

\subsubsection*{森田先生の質問(要約)}
文脈によって意味が変わる単語も存在するか,定義の取得はどのようにしているか.
\subsubsection*{当日の回答}
今回単語はカテゴリ名に付随しており,キーワードという形でアルゴリズムに適用しているため,文脈によって変わるような難しい単語ではないと判断している.

\subsubsection*{シュデシナ先生の質問(要約)}
紙媒体が好きな人もいるが,このシステムを使うユーザーはどのような人か.
\subsubsection*{当日の回答}
紙媒体を使用しているが,IT知識の乏しさによってデジタル化したくてもできない企業が主なターゲットになっている.


\newpage
\thispagestyle{empty}
{\Large
B. 実験で使用したドキュメント
}

\newpage
\addtocontents{toc}{\vspace{1em}}
\addtocontents{toc}{\noindent\textbf{付録 B}\par}
\addtocontents{toc}{\noindent\hspace{4em} 実験で使用したドキュメント \hfill \textbf{B-1}\par}

\renewcommand{\thepage}{B-\arabic{page}}
\setcounter{page}{1}
\chapter*{付録B\\実験で使用したドキュメント}

\setlength{\parindent}{0pt}

\section*{経理}
久保田秀樹, 「会社経理統制令と経理検査」, 山内隆教授退官記念論文集, 第329号, pp. 169-172, 1934年
\\\\
大野巌, 「工業経理規範」, 昭和13年9月, 化学機械, 第2巻 第3号, 1938年
\\\\
柴田雅, 「戦時会社経理統制体制の展開」, The Socio-Economic History Society, 1937年
\\\\
前原真一, 『法人税法の損金経理要件について』, 税務大学校, 研究部教授
\\\\
建部宏明, 「陸軍経理組織の変遷と内部監査制度Ⅱ」, 明治大学経理研究所, 2009年

\section*{人事}
猿田正機, 「トヨタウェイと人事管理・労使関係」, 税務経理協会, 2007年
\\\\
圓生和之, 「公務員の人事異動と人材形成―大卒ホワイトカラーの公民比較からの分析」, 日本労働研究雑誌第759号, pp. 47-53, 2023年
\\\\
津田眞澂, 「新・人事労務管理」, 有斐閣, 1995年
\\\\
津田眞澂編著, 「人事労務管理」, ミネルヴァ書房,1993年
\\\\
平野光俊, 「戦略的パートナーとしての日本の人事部―その役割の本質と課題―」, 神戸大学大学院経営学研究科,2010年

\clearpage
\section*{処務}
所正文, 「一般事務職女性の職業生活意識に関する一考察」, 経営行動科学第3巻第1号, 1988年
\\\\
古武真美, 「女性事務職における派遣労働者の活用」, 近畿大学短期大学論集第44巻第1号, pp. 11-20, 2011年
\\\\
浅海典子, 「女性事務職のキャリア拡大と職場組織」, 日本経済評論社, 2006年
\\\\
寺村絵里子,「女性事務職の賃金と就業行動」,国際短期大学人口学研究第48号,2012年
\\\\
日高義浩,「RPA が事務職に及ぼす影響に関する一考察」,鹿児島経済論集第63巻第1号,2022年

\section*{営業}

\clearpage
\section*{エンジニア}


\newpage
\thispagestyle{empty}
{\Large
C. カテゴライズの指標となる単語群
}

\newpage
\addtocontents{toc}{\vspace{1em}}
\addtocontents{toc}{\noindent\textbf{付録 C}\par}
\addtocontents{toc}{\noindent\hspace{4em} カテゴライズの指標となる単語群 \hfill \textbf{C-1}}

\renewcommand{\thepage}{C-\arabic{page}}
\setcounter{page}{1}
\chapter*{付録C\\カテゴライズの指標となる単語群}

\setlength{\parindent}{0pt}

\section*{経理}
accounting
\\\\
主計
\\\\
会計
\\\\
経理
\\\\
会計学
\\\\
簿記
\\\\
アカウンティング
\\\\
計数
\\\\
accountancy
\\\\
会計業務

\clearpage
\section*{人事}
human resources
\\\\
人事
\\\\
採用
\\\\
雇用
\\\\
配置転換
\\\\
給与
\\\\
労務管理
\\\\
研修
\\\\
人事考課
\\\\
組織開発
\\\\
福利厚生
\\\\
人材育成
\\\\
離職
\\\\
人事評価

\clearpage
\section*{処務}
general affairs
\\\\
庶務
\\\\
総務
\\\\
文書管理
\\\\
資産管理
\\\\
契約管理
\\\\
社内便
\\\\
来客対応
\\\\
ファシリティ
\\\\
備品管理
\\\\
事務補助
\\\\
受付業務
\\\\
オフィスマネジメント

\clearpage
\section*{営業}
sales
\\\\
セールス
\\\\
販売
\\\\
顧客管理
\\\\
販売促進
\\\\
クロージング
\\\\
提案営業
\\\\
営業活動
\\\\
商談
\\\\
新規開拓
\\\\
アポイントメント
\\\\
営業目標
\\\\
受注
\\\\
営業戦略

\clearpage
\section*{エンジニア}
engineer
\\\\
software development
\\\\
システム開発
\\\\
プログラミング
\\\\
ソフトウェア
\\\\
コーディング
\\\\
開発環境
\\\\
テストエンジニア
\\\\
デバッグ
\\\\
ハードウェア
\\\\
ネットワーク
\\\\
技術支援
\\\\
アルゴリズム
\\\\
インフラストラクチャー
\\\\
アーキテクチャ


\end{document}
