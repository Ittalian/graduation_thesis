\chapter{おわりに}
\label{ch:con}

\quad

\section{研究のまとめ}
\label{sec:con_fin}

本研究で開発したシステムは,現在の業務フローにおける課題を解決し得る有効なツールであると結論付けられる.また,さらなる改良を通じて,専門分野や多言語対応,クラウドサービス化といった新たな領域への応用が期待される.
現代では,今後も IT 分野の著しい成長に遅れをとった企業を中心に,さまざまな業務的ニーズが発生すると考えられる.本研究では紙媒体中心業務をさらに細分化した,
ドキュメントのカテゴライズ分野に着目してシステムの作成をしたが,本システムのようなソリューションが生み出されることで,IT 社会の課題が解決されることを願っている.

\section{今後の展望}
\label{sec:con_future}

本システムを現実的な業務フローに組み込むには,ユーザーインターフェース(UI)の最適化も重要である.現状ではシステムのコア機能に重点を置いているが,操作の直感性やユーザーエクスペリエンスの向上が求められる.
例えば,解析結果を可視化するダッシュボードや,フィードバック機能を備えたインターフェースを実装することで,利用者の利便性を高めることが可能である.
処理の精度について,経理,人事,処務といったカテゴリにおいて,高い精度での解析が可能であることが確認された.本システムはさらに分野ごとに特化した最適化を施すことで,
専門分野への応用を推進できる,例えば医療分野や法律分野では特有の専門用語や文書構造が存在するため,分野別にカスタマイズされた解析アルゴリズムが必要である.
また,本システムをクラウドベースで提供することにより,複数のユーザーが同時に利用できる環境を構築し,スケーラビリティを確保することができる.また,APIを公開することで,他のアプリケーションやサービスとの統合を可能にし,システムの拡張性をさらに高めることも有効である.
