\section*{経理}

\begin{itemize}
  \item 久保田秀樹, 「会社経理統制令と経理検査」, 山内隆教授退官記念論文集, 第329号, pp. 169-172, 1934年
  \item 大野巌, 「工業経理規範」, 昭和13年9月, 化学機械, 第2巻 第3号, 1938年
  \item 柴田雅, 「戦時会社経理統制体制の展開」, The Socio-Economic History Society, 1937年
  \item 前原真一, 『法人税法の損金経理要件について』, 税務大学校, 研究部教授
  \item 建部宏明, 「陸軍経理組織の変遷と内部監査制度Ⅱ」, 明治大学経理研究所, 2009年
\end{itemize}

\section*{人事}

\begin{itemize}
  \item 猿田正機, 「トヨタウェイと人事管理・労使関係」, 税務経理協会, 2007年
  \item 圓生和之, 「公務員の人事異動と人材形成―大卒ホワイトカラーの公民比較からの分析」, 日本労働研究雑誌第759号, pp. 47-53, 2023年
  \item 津田眞澂, 「新・人事労務管理」, 有斐閣, 1995年
  \item 津田眞澂編著, 「人事労務管理」, ミネルヴァ書房,1993年
  \item 平野光俊, 「戦略的パートナーとしての日本の人事部―その役割の本質と課題―」, 神戸大学大学院経営学研究科,2010年
\end{itemize}

\clearpage
\section*{処務}

\begin{itemize}
  \item 所正文, 「一般事務職女性の職業生活意識に関する一考察」, 経営行動科学第3巻第1号, 1988年
  \item 古武真美, 「女性事務職における派遣労働者の活用」, 近畿大学短期大学論集第44巻第1号, pp. 11-20, 2011年
  \item 浅海典子, 「女性事務職のキャリア拡大と職場組織」, 日本経済評論社, 2006年
  \item 寺村絵里子,「女性事務職の賃金と就業行動」,国際短期大学人口学研究第48号,2012年
  \item 日高義浩,「RPA が事務職に及ぼす影響に関する一考察」,鹿児島経済論集第63巻第1号,2022年
\end{itemize}
